\documentclass[10pt, a4paper]{report}
\usepackage[utf8]{inputenc}
\usepackage{amsmath}
\usepackage{amsfonts}
\usepackage{amssymb}
\begin{document}

\pagebreak
\tableofcontents
\pagebreak

\chapter{Modo texto}

\section{Negrita, cursiva y subrayado}

Hola desde Latex con
una {\bf palabra} en negrita
{\it esto es cursiva} y
por ultimo
\underline {un texto subrayado} con mas texto

\section{Tamaño letra}

\subsection{Tamaños pequeños}
{\tiny peque \\}
{\normalsize normal \\}
\subsection{Tamaños grandes}
{\large grande \\}
{\LARGE mas grande \\}
{\Huge ENORME \\}

\chapter{Modo matemático}

\section{Modo en línea}

Voy a meter un exponente $^{200} $

Voy a meter una fracción $ a^{\frac{8}{100}} $
$ a^{n^{m^{p}}} $

integrales $ \int_{0}^{10} \cos^2 x dx $

Una línea $ \int_{0}^{10} \frac{3x^2+x}{5x^2} x dx $ de integrales

Expresion matemática en linea aparte:
$$ \int_{0}^{10} \frac{3x^2+x}{5x^2} x dx $$

Expresion matemática más grande
$$ \int_{0}^{10} \frac{\frac{3x^2+x}{5x^2} x dx}{\frac{7x^3+x^2}{9x^2+6}} $$

una llave en horizontal
$$ \underbrace
		{
		\int_{0}^{10}
			\frac{3x^2+x}{5x^2}dx
		}
		_{x^2}
$$

$$\underbrace{ 
	\int_{0}^{10} 
		\frac{3x^2+x}{5x^2} dx 
	+ \int_{0}^{10} 
		\frac{3x^2}{5x^2}
	}_{x^4}
$$

$$\underbrace{ 
	\overbrace{
		\int_{0}^{10} 
			\frac{3x^2+x}{5x^2} dx
			}
			+
	\overbrace{ 
		 \int_{0}^{10} 
			\frac{3x^2}{5x^2}
			}
		}_{x^4}
$$
\end{document}